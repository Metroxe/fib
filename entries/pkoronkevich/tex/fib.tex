\documentclass[11pt]{article}
\usepackage[fleqn]{amsmath}

\newcommand{\one}{\lambda f . \lambda x . f \ x}
\newcommand{\two}{\lambda f . \lambda x . f \ (f \ x)}
\newcommand{\tru}{\lambda x . \lambda y . x}
\newcommand{\flse}{\lambda x . \lambda y . y}
\newcommand{\isz}{\lambda n . n \ (\lambda c . \flse) \ \tru}
\newcommand{\ife}{\lambda p . \lambda a . \lambda b . p \ a \ b}
\newcommand{\suc}{\lambda n . \lambda f . \lambda x . f \ (n \ f \ x)}
\newcommand{\plus}{\lambda m . \lambda n . m \ (\suc) \ n}
\newcommand{\pred}{\lambda n . \lambda f . \lambda x . n \ (\lambda g . \lambda h . h \ (g \ f)) \ (\lambda u . x) \ (\lambda u . u)}
\newcommand{\sub}{\lambda m . \lambda n . n \ (\pred) \ m}
\newcommand{\fix}{\lambda g . (\lambda x . g \ (x \ x)) \ (\lambda x . g \ (x \ x))}
\newcommand{\appo}[2]{#1 \ #2}
\newcommand{\appt}[3]{#1 \ #2 \ #3}

\begin{document}
\begin{gather*}
  \fix \\                                   % recursive function application
  \quad \lambda r . \lambda n . (\ife) \\   % of fib, which takes itself and n, if expression
  \quad \ \appo{(\isz)}{n} \\               % is n zero
  \quad \ (\one) \\                         % if so, return 1
  \quad \ (\plus) \\                        % if not, add 
  \qquad \appo{r}{(\appo{(\pred)}{n})} \\   % the first recursive call, n - 1, to
  \qquad \appo{r}{(\appt{(\sub)}{n}{\two})} % the second recursive call, n - 2
\end{gather*}
\end{document}
